% TODO: scrivere specifica sui dati
\section{Specifica sui dati}
Come visto già nella specifica informale, è necessario poter gestire i seguenti dati:
\subsection{Sensori}
\begin{description}
\item[ID] Il codice identificativo del singolo sensore
\item[Stato] Lo stato del sensore, così da poter gestire possibili
  errori di funzionamento
\item[Posizione] La posizione del sensore in coordinate classiche
  (latitudine, longitudine, altitudine)
\item[Ultimo Controllo] La data dell'ultimo controllo a cui è stato
  sottoposto il sensore, sempre per gestire anomalie
\item[Tipo] Il tipo di sensore
\item[Data di Installazione] La data di posizionamento e installazione
  del sensore
\end{description}
\subsection{Robot}
\begin{description}
\item[Codice] Il codice identificativo del singolo sensore
\item[Tipologia] Il tipo di robot utilizzato
\end{description}
\subsection{Missione}
\begin{description}
\item[Codice] Codice di identificazione univoca della missione
\item[Stato Attuale] Lo stato corrente della missione
\item[Obbiettivo] L'obbiettivo della missione
\item[Data Fine] La data in cui la missione è stata completata
\item[Data Inizio] La data di inizio della missione
\end{description}
\subsection{Report}
\begin{description}
\item[Stato] Lo stato del report
\end{description}
\subsection{Membro}
\begin{description}
\item[Codice] Codice di identificazione del membro d'equipaggio
\item[Nome e Cognome] I dati anagrafici della persona
\item[Ruolo] Il ruolo all'interno dell'equipaggio
\end{description}
\subsection{Anomalia}
\begin{description}
\item[Causa] La causa (certa o presunta) dell'anomalia
\item[Data e Ora] La date e l'ora del rilevamento dell'anomalia
\item[Livello] Il livello di gravità dell'anomalia
\end{description}
\subsection{Intervento}
\begin{description}
\item[Codice] Codice identificativo del singolo intervento
\item[Esito] L'esito dell'intervento
\item[Data] La data prefissata per l'esecuzione dell'intervento
\item[Descrizione] Il tipo dell'intervento
\end{description}
\subsection{Rilevazione}
\begin{description}
\item[Data e Ora] La data e ora del rilevamento
\item[Valore] Il valore rilevato
\end{description}
