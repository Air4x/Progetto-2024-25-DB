\section{Dimensionamento dei dati}
% TODO:Completare inserendo in numero di tuple
Per andare fare una stima sul volume che adranno ad occupare i vari dati in base allla loro tipologia, ci baseremo su:
\begin{itemize}
\item Tabelle/Schemi ottenute in fase di progettazione logica (sezione 3.2)
\item Il numero di missioni previste dall'agenzia spaziale
\end{itemize}
\subsection{Anomalia}
\begin{tabular}{|c|c|c|}
  \hline
  \multicolumn{3}{|c|}{\textbf{ANOMALIA}}\\
  \hline
  Attributo & Tipo & Spazio[BYTE] \\
  \hline
  data & DATE & 7 \\
  ora & DATE & 7 \\
  causa & VARCHAR(30) & 30 \\
  livello & INTEGGER & 4 \\
  sensore & INTEGER & 4 \\
  \hline
\end{tabular}
\begin{equation}
  \begin{aligned}
    &Danomalia=(Ddata + Dora +Dcausa +Dlivello + Dsensore) * Nanomalie=\\
    &=(7+7+30+4+4)*x*10^6=x*49.6  \text{\textbf{MB}}\\
  \end{aligned}
\end{equation}
\subsection{Interventi}
\begin{tabular}{ |c|c|c|}
  \hline
  \multicolumn{3}{|c|}{\textbf{INTERVENTI}} \\
  \hline
  Atributo & Tipo & Spazio[BYTE] \\
  \hline
  codice & INTEGER & 4 \\
  esito & VARCHAR(20) & 20 \\
  calibrazione & VARCHAR(20) & 20 \\
  riparazione & VARCHAR(20) & 20 \\
  sostituzione & VARCHAR(20) & 20 \\
  dataPrefissata & DATE & 7 \\
  \hline
\end{tabular}
\begin{equation}
  \begin{aligned}
    &Dinterventi =\\
    &=(Dcodice + Desito +Dcalibrazione +Driparazione + Dsostituzione + DdataPrefissata) * Nint =\\
    &=(4+20+20+20+20+7)*x*10^6= \textbf{MB}
  \end{aligned}
\end{equation}
\subsection{Membri}
\begin{tabular}{ |c|c|c|}
  \hline
  \multicolumn{3}{|c|}{\textbf{MEMBRI}}\\
  \hline
  Atributo & Tipo & Spazio[BYTE] \\
  \hline
  codice & INTEGER & 4 \\
  nome & VARCHAR(30) & 30 \\
  cognome & VARCHAR(30) & 30 \\
  ruolo & VARCHAR(30) & 30\\
  \hline
\end{tabular}
\begin{equation}
  \begin{aligned}
    &Dmembri =\\
    &=(Dcodice + Dnome +Dcognome +Druolo) * Nmembri =\\
    &=(4+30+30+30) *x*10^6= \textbf{MB}
  \end{aligned}
\end{equation}
\subsection{Missioni}
\begin{tabular}{ |c|c|c|}
  \hline
  \multicolumn{3}{|c|}{\textbf{MISSIONI}}\\
  \hline
  Atributo & Tipo & Spazio[BYTE] \\
  \hline
  codice & INTEGER & 4 \\
  dataFine & DATE & 7 \\
  obbiettivo & VARCHAR(100) & 100 \\
  dataInizio & DATE & 7 \\
  statoAttuale  & VARCHAR(30) & 30 \\
  dataFine & VARCHAR(30) & 30 \\
  \hline
\end{tabular}
\begin{equation}
  \begin{aligned}
    &Dmissioni =\\
    &=(Dcodice+DdataFine+Dobbietivo+DdataInizio+DstatoA+DdataFine)*Nmiss=\\
    &=(4+7+100+7+30+30)*x*10^6= \textbf{MB}
  \end{aligned}
\end{equation}
\subsection{Report}
\begin{tabular}{ |c|c|c|}
  \hline
  \multicolumn{3}{|c|}{\textbf{REPORT}}\\
  \hline
  Atributo & Tipo & Spazio[BYTE] \\
  \hline
  stato & VARCHAR(30) & 30 \\
  autore & VARCHAR(30) & 30\\
  \hline
\end{tabular}

\begin{equation}
  \begin{aligned}
    &Dreport =\\
    &=(Dstato+Dautore)*Nreport=\\
    &=(30+30)*x*10^6= \textbf{MB}
  \end{aligned}
\end{equation}
\subsection{Rilevazioni}
\begin{tabular}{ |c|c|c|}
  \hline
  \multicolumn{3}{|c|}{\textbf{RILEVAZIONI}}\\
  \hline
  Atributo & Tipo & Spazio[BYTE] \\
  \hline
  data & DATE & 7 \\
  ora & DATE &  7 \\
  valore & INTEGER & 4 \\
  sensore & INTEGER & 4 \\
  \hline
\end{tabular}
\begin{equation}
  \begin{aligned}
    &Drivelazione =\\
    &=(Ddata+Dora+valore+Dsensore)*Nrivelazione=\\
    &=(7+7+4+4+)*x*10^6= \textbf{MB}
  \end{aligned}
\end{equation}
\subsection{Robot}
\begin{tabular}{ |c|c|c|}
  \hline
  \multicolumn{3}{|c|}{\textbf{ROBOT}}\\
  \hline
  Atributo & Tipo & Spazio[BYTE] \\
  \hline
  codice & INTEGER & 4\\
  tipologia & VARCHAR(30) & 30\\
  \hline
\end{tabular}
\begin{equation}
  \begin{aligned}
    &Drobot =\\
    &=(Dcodice+Dtipologia)*Nrobot=\\
    &=(4+30)*x*10^6= \textbf{MB}
  \end{aligned}
\end{equation}
\subsection{Sensori}
\begin{tabular}{ |c|c|c|}
  \hline
  \multicolumn{3}{|c|}{\textbf{SENSORI}}\\
  \hline
  Atributo & Tipo & Spazio[BYTE] \\
  \hline
  ID & INTEGER & 4 \\
  stato & VARCHAR(30) & 30\\
  dataInstallazione & DATE & 7\\
  tipo & VARCHAR(30) & 30\\
  ultimoControllo & DATE & 7\\
  latitudine & DECIMAL(10,4) & 9\\
  altitudine & DECIMAL(10,4) & 9\\
  longitudine & DECIMAL(10,4) & 9\\
  \hline
\end{tabular}
\begin{equation}
  \begin{aligned}
    &Dsensore =\\
    &=(Did+Dstato+DdataInst+DultCont+Dlat+Dalt+long)*Nsen=\\
    &=(4+30+7+30+7+9+9+9)*x*10^6= \textbf{MB}
  \end{aligned}
\end{equation}
\subsection{Gestione dello spazio e dell'affidabilità}
Dalle stime fatte precedentemente possiamo stimare un volume di informazioni pari:
\begin{equation}
  Dtot=X\\
\end{equation}
