\section{Dimensionamento dei dati}
% TODO:Completare inserendo in numero di tuple
Per andare fare una stima sul volume che adranno ad occupare i vari dati in base allla loro tipologia, ci baseremo su:
\begin{itemize}
\item Tabelle/Schemi ottenute in fase di progettazione logica (sezione 3.2)
\item Il numero di missioni previste dall'agenzia spaziale
\end{itemize}
In questo caso ci baseremo per sul popolamento fatto nella sezione 4.4 per fare il dimensionamento.
\subsection{Anomalia}
\begin{tabular}{|c|c|c|}
  \hline
  \multicolumn{3}{|c|}{\textbf{ANOMALIA}}\\
  \hline
  Attributo & Tipo & Spazio(Byte) \\
  \hline
  data & DATE & 7 \\
  ora & VARCHAR(8)  & 8 \\
  causa & VARCHAR(30) & 30 \\
  livello & INTEGGER & 4 \\
  sensore & INTEGER & 4 \\
  \hline
\end{tabular}
\begin{equation}
  \begin{aligned}
    D_{\text{anomalia}}&=(D_{\text{data}} + D_{\text{ora}} +D_{\text{causa}} +D_{\text{livello}} + D_{\text{sensore}}) \cdot N_{\text{anomalie}}=\\
    &=(7+8+30+4+4)\cdot 6 = 318\text{\textbf{B}}\\
  \end{aligned}
\end{equation}
\subsection{Interventi}
\begin{tabular}{ |c|c|c|}
  \hline
  \multicolumn{3}{|c|}{\textbf{INTERVENTI}} \\
  \hline
  Attributo & Tipo & Spazio(Byte) \\
  \hline
  codice & INTEGER & 4 \\
  esito & VARCHAR(20) & 20 \\
  calibrazione & VARCHAR(20) & 20 \\
  riparazione & VARCHAR(20) & 20 \\
  sostituzione & VARCHAR(20) & 20 \\
  dataPrefissata & DATE & 7 \\
  \hline
\end{tabular}
\begin{equation}
  \begin{aligned}
    D_{\text{interventi}} &=(D_{\text{codice}} + D_{\text{esito}} +D_{\text{calibrazione}} +D_{\text{riparazione}} + D_{\text{sostituzione}} + D_{text{dataPrefissata}}) \cdot  Nint =\\
    &=(4+20+20+20+20+7)\cdot 5 = 455\textbf{B}
  \end{aligned}
\end{equation}
\subsection{Membri}
\begin{tabular}{ |c|c|c|}
  \hline
  \multicolumn{3}{|c|}{\textbf{MEMBRI}}\\
  \hline
  Attributo & Tipo & Spazio(Byte) \\
  \hline
  codice & INTEGER & 4 \\
  nome & VARCHAR(30) & 30 \\
  cognome & VARCHAR(30) & 30 \\
  ruolo & VARCHAR(30) & 30\\
  \hline
\end{tabular}
\begin{equation}
  \begin{aligned}
    D_{\text{membri}} &=(D_{\text{codice}} + D_{\text{nome}} +D_{\text{cognome}} +D_{\text{ruolo}}) \cdot  N_{\text{membri}} =\\
    &=(4+30+30+30) \cdot 5 = 470\textbf{B}
  \end{aligned}
\end{equation}
\subsection{Missioni}
\begin{tabular}{ |c|c|c|}
  \hline
  \multicolumn{3}{|c|}{\textbf{MISSIONI}}\\
  \hline
  Attributo & Tipo & Spazio(Byte) \\
  \hline
  codice & INTEGER & 4 \\
  dataFine & DATE & 7 \\
  obbiettivo & VARCHAR(100) & 100 \\
  dataInizio & DATE & 7 \\
  statoAttuale  & VARCHAR(30) & 30 \\
  resoconto & VARCHAR(30) & 30 \\
  \hline
\end{tabular}
\begin{equation}
  \begin{aligned}
    &D_{\text{missioni}} =\\
    &=(D_{\text{codice}}+D_{\text{dataFine}}+D_{\text{obbietivo}}+D_{\text{dataInizio}}+D_{\text{statoA}}+D_{\text{dataFine}})\cdot N_{\text{miss}}=\\
    &=(4+7+100+7+30+30)\cdot 5 = 890\textbf{B}
  \end{aligned}
\end{equation}
\subsection{Risorsa_1}
\begin{tabular}{|c|c|c|}
  \hline
  \multicolumn{3}{|c|}{\textbf{RISORSA_1}}\\
  \hline
  Attributo & Tipo & Spazio(Byte) \\
  \hline
  sensore & INTEGGER & 4 \\
  missione & INTEGER & 4 \\
  \hline
\end{tabular}
\begin{equation}
  \begin{aligned}
    D_{\text{risorsa_1}}&=(D_{\text{sensore}} + D_{\text{missione}}) \cdot N_{\text{risorse_1}}=\\
    &=(4+4)\cdot 5 = 40\text{\textbf{B}}\\
  \end{aligned}
\end{equation}
\subsection{Report}
\begin{tabular}{|c|c|c|}
  \hline
  \multicolumn{3}{|c|}{\textbf{REPORT}}\\
  \hline
  Attributo & Tipo & Spazio(Byte) \\
  \hline
  stato & VARCHAR(30) & 30 \\
  autore & INTEGER & 4\\
  \hline
\end{tabular}
\begin{equation}
  \begin{aligned}
    D_{\text{report}} &=(D_{\text{stato}}+D_{\text{autore}})\cdot N_{\text{report}}=\\
    &=(30+4)\cdot 5 = 170\textbf{B}
  \end{aligned}
\end{equation}
\subsection{Rilevazioni}
\begin{tabular}{|c|c|c|}
  \hline
  \multicolumn{3}{|c|}{\textbf{RILEVAZIONI}}\\
  \hline
  Attributo & Tipo & Spazio(Byte) \\
  \hline
  data & DATE & 7 \\
  ora & VARCHAR(8)  & 8\\
  valore & INTEGER & 4 \\
  sensore & INTEGER & 4 \\
  \hline
\end{tabular}
\begin{equation}
  \begin{aligned}
    D_{\text{rivelazione}} &=\\
    &=(D_{\text{data}}+D_{\text{ora}}+D_{\text{valore}}+D_{\text{sensore}})\cdot N_{\text{rivelazione}}=\\
    &=(7+8+4+4)\cdot 5 = 115\textbf{B}
  \end{aligned}
\end{equation}
\subsection{Robot}
\begin{tabular}{ |c|c|c|}
  \hline
  \multicolumn{3}{|c|}{\textbf{ROBOT}}\\
  \hline
  Attributo & Tipo & Spazio(Byte) \\
  \hline
  codice & INTEGER & 4\\
  tipologia & VARCHAR(30) & 30\\
  \hline
\end{tabular}
\begin{equation}
  \begin{aligned}
    D_{\text{robot}} &=(D_{\text{codice}}+D_{\text{tipologia}})\cdot N_{\text{robot}}=\\
    &=(4+30)\cdot 5 = 170\textbf{B}
  \end{aligned}
\end{equation}
\subsection{Sensori}
\begin{tabular}{ |c|c|c|}
  \hline
  \multicolumn{3}{|c|}{\textbf{SENSORI}}\\
  \hline
  Attributo & Tipo & Spazio(Byte) \\
  \hline
  ID & INTEGER & 4 \\
  stato & VARCHAR(30) & 30\\
  dataInstallazione & DATE & 7\\
  tipo & VARCHAR(30) & 30\\
  ultimoControllo & DATE & 7\\
  latitudine & DECIMAL(10,4) & 9\\
  altitudine & DECIMAL(10,4) & 9\\
  longitudine & DECIMAL(10,4) & 9\\
  \hline
\end{tabular}
\begin{equation}
  \begin{aligned}
    D_{\text{sensore}} &=\\
    &=(D_{\text{id}}+D_{\text{stato}}+D_{\text{dataInst}}+D_{\text{ultCont}}+D_{\text{lat}}+D_{\text{alt}}+D_{\text{long}})\cdot N_{\text{sen}}=\\
    &=(4+30+7+30+7+9+9+9)\cdot 5 = 525\textbf{B}
  \end{aligned}
\end{equation}
\subsection{Risorsa_2}
\begin{tabular}{|c|c|c|}
  \hline
  \multicolumn{3}{|c|}{\textbf{RISORSA_2}}\\
  \hline
  Attributo & Tipo & Spazio(Byte) \\
  \hline
  robot & INTEGGER & 4 \\
  missione & INTEGER & 4 \\
  \hline
\end{tabular}
\begin{equation}
  \begin{aligned}
    D_{\text{risorsa_2}}&=(D_{\text{robot}} + D_{\text{missione}}) \cdot N_{\text{risorse_2}}=\\
    &=(4+4)\cdot 6 = 48\text{\textbf{B}}\\
  \end{aligned}
\end{equation}
\subsection{Gestione dello spazio e dell'affidabilità}
Dalle stime fatte precedentemente possiamo stimare un volume di informazioni pari:
\begin{equation}
  D_{\text{tot}}=(318+455+470+890+40+170+115+170+525+48)= 3201\text{\textbf{B}}\\
\end{equation}

%%% Local Variables:
%%% mode: LaTeX
%%% TeX-master: "Tesina"
%%% End: