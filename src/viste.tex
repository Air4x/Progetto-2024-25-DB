\section{Viste}
\begin{description}
\item{Vista dei sensori attivi con ultima rilevazione}
Creare una vista che unisca le informazioni sui sensori attivi con la loro ultima rilevazione registrata.
\begin{minted}{sql}
CREATE VIEW SENSORI_ATTIVI_RILEVAZIONI AS
SELECT SENSORI.ID AS SENSORI, RILEVAZIONI.DATA AS ULTIME_RILEVAZIONI
FROM SENSORI JOIN RILEVAZIONI ON SENSORI.ID = RILEVAZIONI.SENSORE
WHERE SENSORI.STATO='Attivo' AND RILEVAZIONI.DATA = (
    SELECT MAX(RILEVAZIONI.DATA)
    FROM RILEVAZIONI
    WHERE SENSORI.ID=RILEVAZIONI.SENSORE
);
\end{minted}
\item{Vista degli interventi completati e il loro esito}
Mostrare tutti gli interventi completati con dettagli su calibrazione, riparazione e sostituzione.
\begin{minted}{sql}
CREATE VIEW INTERVENTI_COMPLETATI AS
SELECT INTERVENTI.CODICE AS INTERVENTI,INTERVENTI.ESITO,INTERVENTI.CALIBRAZIONE,INTERVENTI.RIPARAZIONE,INTERVENTI.SOSTITUZIONE
FROM INTERVENTI
WHERE INTERVENTI.ESITO = 'Completato';
\end{minted}
\item{Vista dei membri e dei report creati}
Unire la tabella MEMBRI con REPORT per ottenere un elenco dei membri che hanno scritto report e il loro stato.
\begin{minted}{sql}
CREATE VIEW MEMBRI_REPORT AS
SELECT MEMBRI.CODICE AS MEMBRO_CODE, REPORT.STATO AS STATO_REPORT
FROM MEMBRI JOIN REPORT ON MEMBRI.CODICE = REPORT.AUTORE;
\end{minted}
\item{Vista delle missioni con i robot assegnati}
Creare una vista che visualizzi le missioni attive e i robot assegnati tramite RISORSA_1.
\begin{minted}{sql}
CREATE VIEW MISSIONI_ROBOT AS
SELECT MISSIONI.CODICE AS MISSIONI, RISORSA_1.ROBOT
FROM MISSIONI JOIN RISORSA_1 ON MISSIONI.CODICE = RISORSA_1.MISSIONE
WHERE MISSIONI.STATOATTUALE = 'Attiva'; 
\end{minted}
\item{Vista delle anomalie recenti}
Mostrare tutte le anomalie registrate nell’ultimo mese con i dettagli del sensore coinvolto.
\begin{minted}{sql}
CREATE VIEW ANOMALIE_RECENTI AS
SELECT SENSORI.ID AS SENSORI, ANOMALIE.DATA AS DATA_ANOMALIA,SENSORI.LATITUDINE ,SENSORI.LONGITUDINE,SENSORI.ALTITUDINE
FROM SENSORI JOIN ANOMALIE ON SENSORI.ID = ANOMALIE.SENSORE
WHERE TO_DATE(ANOMALIE.DATA, 'YYYY-MM-DD') >= SYSDATE - 30
\end{minted}
\end{description}
%%% Local Variables:
%%% mode: LaTeX
%%% TeX-master: "Tesina"
%%% End: