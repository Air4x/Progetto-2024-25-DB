\section{Query}
\subsection{Recupero anomalie gravi}
Selezionare tutte le anomalie con un livello maggiore o uguale a una certa soglia (es. 4), indicando il sensore coinvolto e la data.
\begin{description}
\begin{minted}{sql}
SELECT SENSORI.ID, ANOMALIE.LIVELLO
FROM SENSORI JOIN ANOMALIE ON SENSORI.ID = ANOMALIE.SENSORE
WHERE ANOMALIE.LIVELLO >= 4
GROUP BY SENSORI.ID,ANOMALIE.LIVELLO;
\end{minted}
\subsection{Monitoraggio degli interventi pianificati}
Recuperare tutti gli interventi con stato "Pianificato" e la data prevista per l'esecuzione.
\begin{minted}{sql}
SELECT INTERVENTI.CODICE, INTERVENTI.ESITO
FROM INTERVENTI
WHERE INTERVENTI.ESITO = 'Pianificato'
GROUP BY INTERVENTI.CODICE, INTERVENTI.ESITO;
\end{minted}
\subsection{Lista delle missioni attive con risorse assegnate}
Visualizzare tutte le missioni con stato "Attiva", mostrando i robot e i sensori associati.
\begin{minted}{sql}
SELECT MISSIONI.STATOATTUALE, RISORSA_1.ROBOT, RISORSA_2.SENSORE
FROM MISSIONI JOIN RISORSA_1 ON MISSIONI.CODICE = RISORSA_1.MISSIONE JOIN RISORSA_2 ON RISORSA_1.MISSIONE = RISORSA_2.MISSIONE  
WHERE MISSIONI.STATOATTUALE = 'Attiva'
GROUP BY MISSIONI.STATOATTUALE, RISORSA_1.ROBOT, RISORSA_2.SENSORE;
\end{minted}
\subsection{Sensori con più anomalie registrate}
Contare il numero di anomalie per ciascun sensore e restituire quelli con il maggior numero di problemi.
\begin{minted}{sql}
CREATE VIEW ANOMALIE_SENSORI AS
SELECT SENSORI.ID AS SENSORI,COUNT(ANOMALIE.SENSORE) AS NUM_ANOMALIE
FROM SENSORI JOIN ANOMALIE ON SENSORI.ID = ANOMALIE.SENSORE
GROUP BY SENSORI.ID;

SELECT SENSORI,NUM_ANOMALIE
FROM ANOMALIE_SENSORI
WHERE NUM_ANOMALIE=(
    SELECT MAX(NUM_ANOMALIE)
    FROM ANOMALIE_SENSORI
);
\end{minted}
\subsection{Ultime rilevazioni per ogni sensore}
Ottenere l'ultima misurazione registrata per ogni sensore.
\begin{minted}{sql}
SELECT R1.SENSORE AS SENSORE,R1.VALORE,R1.DATA
FROM RILEVAZIONI R1
WHERE (DATA,ORA) = (
    SELECT MAX(R2.DATA),MAX(R2.ORA)
    FROM RILEVAZIONI R2
    WHERE R1.SENSORE = R2.SENSORE
    RAISE_APPLICATION_ERROR(-20001,'ERRORE: Sensore non trovato');
);
\end{minted}
\end{description}
%%% Local Variables:
%%% mode: LaTeX
%%% TeX-master: "Tesina"
%%% End: