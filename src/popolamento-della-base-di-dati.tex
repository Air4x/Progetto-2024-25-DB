\section{Popolamento della base di dati}
%TODO: In caso di cambiamentoi nel popolamento ricardare di aggiornare il dimensionamento
In tale sezioni ci occuperemo del popolamento della base dati con l'ausilio di Chatgip per la creazione di dati fittizi con il fini di testare la nostra base dati. In otteniamo:
\begin{itemize}
\item 5 Sensori
\item 6 Anomalie
\item 5 Interventi
\item 5 Membri
\item 5 Report
\item 5 Missioni
\item 5 Rilevazioni
\item 5 Robot
\item 5 Risorsa_1
\item 6 Risorsa_2
\end{itemize}
\subsection{\texttt{Sensori}}
\begin{minted}{sql}
INSERT INTO SENSORI (STATO, DATAINSTALLAZIONE, TIPO, ULTIMOCONTROLLO, LATITUDINE, LONGITUDINE, ALTITUDINE) VALUES
('Attivo', TO_DATE('2023-01-15', 'YYYY-MM-DD'), 'Temperatura', TO_DATE('2024-02-01', 'YYYY-MM-DD'), 45.1234, 12.5678, 100.5),
('Guasto', TO_DATE('2022-06-20', 'YYYY-MM-DD'), 'Pressione', TO_DATE('2024-01-10', 'YYYY-MM-DD'), 44.8765, 13.6789, 200.3),
('Attivo', TO_DATE('2023-03-10', 'YYYY-MM-DD'), 'Umidità', TO_DATE('2024-02-15', 'YYYY-MM-DD'), 46.2345, 11.3456, 150.0),
('Manutenzione', TO_DATE('2023-05-22', 'YYYY-MM-DD'), 'Temperatura', TO_DATE('2024-02-28', 'YYYY-MM-DD'), 45.6789, 12.4567, 180.7),
('Attivo', TO_DATE('2022-11-05', 'YYYY-MM-DD'), 'Pressione', TO_DATE('2024-02-10', 'YYYY-MM-DD'), 46.7890, 13.7890, 220.1);
\end{minted}
\subsection{\texttt{Anomalie}}
\begin{minted}{sql}
INSERT INTO ANOMALIE (DATA, ORA, CAUSA, LIVELLO, SENSORE) VALUES 
(TO_DATE('2024-02-10', 'YYYY-MM-DD'),'10:30:00', 'Sbalzo di temperatura', 3, 1),
(TO_DATE('2024-01-25', 'YYYY-MM-DD'),'08:15:00', 'Guasto elettronico', 5, 2),
(TO_DATE('2024-02-20', 'YYYY-MM-DD'),'14:45:00', 'Errore di calibrazione', 2, 2),
(TO_DATE('2024-02-22', 'YYYY-MM-DD'),'18:30:00', 'Sbalzo di temperatura', 4, 1),
(TO_DATE('2024-02-28', 'YYYY-MM-DD'),'20:00:00','Sensore non risponde', 3, 1),
(TO_DATE('2024-01-09', 'YYYY-MM-DD'),'10:00:00', 'Danno Hardware', 2, 3);
\end{minted}
\subsection{\texttt{Interventi}}
\begin{minted}{sql}
INSERT INTO INTERVENTI (ESITO, CALIBRAZIONE, RIPARAZIONE, SOSTITUZIONE, DATAPREFISSATA) VALUES 
('Completato', 'Sì', 'No', 'No', TO_DATE('2024-02-20', 'YYYY-MM-DD')),
('In corso', 'No', 'Sì', 'No', TO_DATE('2024-03-05', 'YYYY-MM-DD')),
('Pianificato', 'No', 'No', 'Sì', TO_DATE('2024-03-15', 'YYYY-MM-DD')),
('Completato', 'Sì', 'Sì', 'No', TO_DATE('2024-02-25', 'YYYY-MM-DD')),
('Annullato', 'No', 'No', 'No', TO_DATE('2024-04-01', 'YYYY-MM-DD')),
\end{minted}
\subsection{\texttt{Membri}}
\begin{minted}{sql}
INSERT INTO MEMBRI (NOME, COGNOME, RUOLO) VALUES
('Mario', 'Rossi', 'Tecnico'),
('Laura', 'Bianchi', 'Supervisore'),
('Giovanni', 'Verdi', 'Operatore'),
('Elena', 'Neri', 'Analista'),
('Paolo', 'Moretti', 'Ingegnere'),
\end{minted}
\subsection{\texttt{Report}}
\begin{minted}{sql}
INSERT INTO REPORT (STATO, AUTORE) VALUES 
('Confermato', 1),
('In revisione', 2),
('Approvato', 3),
('Respinto', 4),
('In attesa', 5);
\end{minted}
\subsection{\texttt{Missioni}}
\begin{minted}{sql}
INSERT INTO MISSIONI (DATAFINE, OBBIETTIVO, DATAINIZIO, STATOATTUALE, RESOCONTO) VALUES 
(TO_DATE('2024-03-10', 'YYYY-MM-DD'), 'Manutenzione sensori', TO_DATE('2024-02-25', 'YYYY-MM-DD'), 'Attiva', 'Confermato'),
(TO_DATE('2024-04-01', 'YYYY-MM-DD'), 'Installazione nuovi dispositivi', TO_DATE('2024-03-15', 'YYYY-MM-DD'), 'Pianificata', 'In revisione'),
(TO_DATE('2024-05-12', 'YYYY-MM-DD'), 'Riparazione anomalie', TO_DATE('2024-04-20', 'YYYY-MM-DD'), 'In corso', 'Approvato'),
(TO_DATE('2024-06-10', 'YYYY-MM-DD'), 'Test nuovi sensori', TO_DATE('2024-05-01', 'YYYY-MM-DD'), 'Pianificata', 'In attesa'),
(TO_DATE('2024-07-05', 'YYYY-MM-DD'), 'Upgrade firmware', TO_DATE('2024-06-20', 'YYYY-MM-DD'), 'Attiva', 'Confermato');
\end{minted}
\subsection{\texttt{Rilevazioni}}
\begin{minted}{sql}
INSERT INTO RILEVAZIONI (DATA, ORA, VALORE, SENSORE) VALUES 
(TO_DATE('2024-02-12', 'YYYY-MM-DD'),'12:00:00', 25, 1),
(TO_DATE('2024-02-12', 'YYYY-MM-DD'),'12:30:00', 27, 1),
(TO_DATE('2024-02-14', 'YYYY-MM-DD'),'09:45:00', 30, 4),
(TO_DATE('2024-02-15', 'YYYY-MM-DD'),'16:20:00', 22, 3),
(TO_DATE('2024-07-05', 'YYYY-MM-DD'),'19:30:00', 22, 5);
\end{minted}
\subsection{\texttt{Robot}}
\begin{minted}{sql}
INSERT INTO ROBOT (TIPOLOGIA) VALUES ('Ricognizione'),
('Riparazione'),
('Supporto logistico'),
('Monitoraggio'),
('Emergenza');
\end{minted}
\subsection{\texttt{Risorsa 1}}
\begin{minted}{sql}
INSERT INTO RISORSA_1 (ROBOT, MISSIONE) VALUES 
(1, 1),
(2, 2),
(3, 4),
(2, 3),
(4, 5);
\end{minted}
\subsection{\texttt{Risorsa 2}}
\begin{minted}{sql}
INSERT INTO RISORSA_2 (SENSORE, MISSIONE) VALUES 
(1, 1),
(2, 2),
(3, 3),
(4, 5),
(1, 4),
(2, 1);
\end{minted}


%%% Local Variables:
%%% mode: LaTeX
%%% TeX-master: "Tesina"
%%% End: