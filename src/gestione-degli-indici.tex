\section{Gestione degli indici}
\mint{sql}/ /
%TODO: Da completare
Gli indici sono struttura dati ordinata impiegata nel DBMS per la localizzazione di un specifico record andando ad velocizzare le operazioni di ricerca, sono 
caratterizati da una data entry (record memorizato in un file indice) che può essere costituito:
\begin{itemize}
\item Intero record di dati
\item Coppia costituita da una \texttt{chiave} e \texttt{rid}
\item Coppia costituita da una \texttt{chiave} e \texttt{lista di rid}
\end{itemize}
Dove in \texttt{DBMS Oracle} va creare in maniera automati indici di tipo \texttt{B+Tree} per ogni chiave primaria di tutte le tabelle.
\begin{description}
\begin{minted}{sql}
 CREATE INDEX INTERVENTI_CODICE ON INTERVENTI (codice);
 CREATE INDEX REPORT_STATO ON REPORT (stato);
 CREATE INDEX ANOMALIE_DATA_ORA ON ANOMALIE (data, ora);
 CREATE INDEX RIVELAZIONI_DATA_ORA ON RIVELAZIONI (data, ora);
 CREATE INDEX MISSIONI_CODICE ON MISSIONI (codice);
\end{minted}
\end{description}

%%% Local Variables:
%%% mode: LaTeX
%%% TeX-master: "Tesina"
%%% End: