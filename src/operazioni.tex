\chapter{Operazioni}
Per l'implementazione di funzionalità che la base dati deve offrire agli utento aviene mediante l'utilizzo di:
\begin{itemize}
\item Viste, sono delle tabelle, che vengono descritte in termini di altre tabelle, una \textbf{relazione virtuale}, dove sue tuple non sono memorizzate nella base dati ma ricavabili medianti interrogazioni, deliniando un iterfaccia messa a disposizione di utenti e applicazioni. La sua utilita si trova nei casi in cui si debbano vare tabelle che mettono in relazione campi di tabelle “distanti” nel modello logico.
\item Query, per verificare il corretto funzionamento della base di dati.
\item Trigger, sono procedure che si attivano quando vengono eseguite ogni volta che si fa un operazione stabilitasu un oggetto specifico della base dati.
\end{itemize}
\section{Viste}
\subsection{Vista dei sensori attivi con ultima rilevazione}
Creare una vista che unisca le informazioni sui sensori attivi con la loro ultima rilevazione registrata.
\begin{minted}{sql}
CREATE VIEW SENSORI_ATTIVI_RILEVAZIONI AS
SELECT SENSORI.ID AS SENSORI, RILEVAZIONI.DATA AS ULTIME_RILEVAZIONI
FROM SENSORI JOIN RILEVAZIONI ON SENSORI.ID = RILEVAZIONI.SENSORE
WHERE SENSORI.STATO='Attivo' AND RILEVAZIONI.DATA = (
    SELECT MAX(RILEVAZIONI.DATA)
    FROM RILEVAZIONI
    WHERE SENSORI.ID=RILEVAZIONI.SENSORE
);
\end{minted}
\subsection{Vista degli interventi completati e il loro esito}
Mostrare tutti gli interventi completati con dettagli su calibrazione, riparazione e sostituzione.
\begin{minted}[breaklines]{sql}
CREATE VIEW INTERVENTI-COMPLETATI AS
SELECT INTERVENTI.CODICE AS INTERVENTI,INTERVENTI.ESITO,INTERVENTI.CALIBRAZIONE,
->INTERVENTI.RIPARAZIONE,INTERVENTI.SOSTITUZIONE
FROM INTERVENTI
WHERE INTERVENTI.ESITO = 'Completato';
\end{minted}
\subsection{Vista dei membri e dei report creati}
Unire la tabella MEMBRI con REPORT per ottenere un elenco dei membri che hanno scritto report e il loro stato.
\begin{minted}{sql}
CREATE VIEW MEMBR_REPORT AS
SELECT MEMBRI.CODICE AS MEMBRO_CODE, REPORT.STATO AS STATO_REPORT
FROM MEMBRI JOIN REPORT ON MEMBRI.CODICE = REPORT.AUTORE;
\end{minted}
\subsection{Vista delle missioni con i robot assegnati}
Creare una vista che visualizzi le missioni attive e i robot assegnati tramite RISORSA\_1.
\begin{minted}{sql}
CREATE VIEW MISSIONI-ROBOT AS
SELECT MISSIONI.CODICE AS MISSIONI, RISORSA_1.ROBOT
FROM MISSIONI JOIN RISORSA_1 ON MISSIONI.CODICE = RISORSA_1.MISSIONE
WHERE MISSIONI.STATOATTUALE = 'Attiva'; 
\end{minted}
\subsection{Vista delle anomalie recenti}
Mostrare tutte le anomalie registrate nell’ultimo mese con i dettagli del sensore coinvolto.
\begin{minted}[breaklines]{sql}
CREATE VIEW ANOMALIE_RECENTI AS
SELECT SENSORI.ID AS SENSORI, ANOMALIE.DATA AS DATA_ANOMALIA,SENSORI.LATITUDINE ,SENSORI.LONGITUDINE,SENSORI.ALTITUDINE
FROM SENSORI JOIN ANOMALIE ON SENSORI.ID = ANOMALIE.SENSORE
WHERE TO_DATE(ANOMALIE.DATA, 'YYYY-MM-DD') >= SYSDATE - 30
\end{minted}

%%% Local Variables:
%%% mode: LaTeX
%%% TeX-master: "Tesina"
%%% End:
\section{Query}
\subsection{Recupero anomalie gravi}
Selezionare tutte le anomalie con un livello maggiore o uguale a una certa soglia (es. 4), indicando il sensore coinvolto e la data.
\begin{minted}{sql}
SELECT SENSORI.ID, ANOMALIE.LIVELLO
FROM SENSORI JOIN ANOMALIE ON SENSORI.ID = ANOMALIE.SENSORE
WHERE ANOMALIE.LIVELLO >= 4
GROUP BY SENSORI.ID,ANOMALIE.LIVELLO;
\end{minted}
\subsection{Monitoraggio degli interventi pianificati}
Recuperare tutti gli interventi con stato "Pianificato" e la data prevista per l'esecuzione.
\begin{minted}{sql}
SELECT INTERVENTI.CODICE, INTERVENTI.ESITO
FROM INTERVENTI
WHERE INTERVENTI.ESITO = 'Pianificato'
GROUP BY INTERVENTI.CODICE, INTERVENTI.ESITO;
\end{minted}
\subsection{Lista delle missioni attive con risorse assegnate}
Visualizzare tutte le missioni con stato "Attiva", mostrando i robot e i sensori associati.
\begin{minted}[breaklines]{sql}
SELECT MISSIONI.STATOATTUALE, RISORSA_1.ROBOT, RISORSA-2.SENSORE
FROM MISSIONI JOIN RISORSA-1 ON MISSIONI.CODICE = RISORSA-1.MISSIONE JOIN RISORSA_2 ON RISORSA_1.MISSIONE = RISORSA_2.MISSIONE  
WHERE MISSIONI.STATOATTUALE = 'Attiva'
GROUP BY MISSIONI.STATOATTUALE, RISORSA_1.ROBOT, RISORSA_2.SENSORE;
\end{minted}
\subsection{Sensori con più anomalie registrate}
Contare il numero di anomalie per ciascun sensore e restituire quelli con il maggior numero di problemi.
\begin{minted}{sql}
CREATE VIEW ANOMALIA_SENSORI AS
SELECT SENSORI.ID AS SENSORI,COUNT(ANOMALIE.SENSORE) AS NUM_ANOMALIE
FROM SENSORI JOIN ANOMALIE ON SENSORI.ID = ANOMALIE.SENSORE
GROUP BY SENSORI.ID;

SELECT SENSORI,NUM_ANOMALIE
FROM ANOMALIE-SENSORI
WHERE NUM_ANOMALIE=(
    SELECT MAX(NUM_ANOMALIE)
    FROM ANOMALIE_SENSORI
);
\end{minted}
\subsection{Ultime rilevazioni per ogni sensore}
Ottenere l'ultima misurazione registrata per ogni sensore.
\begin{minted}{sql}
SELECT R1.SENSORE AS SENSORE,R1.VALORE,R1.DATA
FROM RILEVAZIONI R1
WHERE (DATA,ORA) = (
    SELECT MAX(R2.DATA),MAX(R2.ORA)
    FROM RILEVAZIONI R2
    WHERE R1.SENSORE = R2.SENSORE
    RAISE_APPLICATION_ERROR(-20001,'ERRORE: Sensore non trovato');
);
\end{minted}
%%% Local Variables:
%%% mode: LaTeX
%%% TeX-master: "Tesina"
%%% End:
\section{Trigger}
\begin{description}
\subsection{Trigger per aggiornare automaticamente la data dell'ultimo controllo di un sensore}
Quando viene registrata una nuova rilevazione per un sensore, aggiornare il campo ULTIMOCONTROLLO.
\begin{minted}{sql}
CREATE OR REPLACE TRIGGER UPDATE_ULTIMO_CONTROLLO
AFTER INSERT ON RIVELAZIONI
FOR EACH ROW
BEGIN
    UPDATE SENSORI
    SET ULTIMOCONTROLLO = :NEW.DATA
    WHERE ID = :NEW.SENSORE
END;
\end{minted}
\subsection{Trigger per impedire l'inserimento di anomalie con data futura}
Controllare che la data delle anomalie non sia successiva alla data odierna.
\begin{minted}{sql}
CREATE OR REPLACE TRIGGER ANOMALIE_FUTURE
BEFORE INSERT ON ANOMALIE
FOR EACH ROW
DECLARE 
     DATA_FUTURA EXCEPTION;
BEGIN 
IF :NEW.DATA > SYSDATE THEN
        RAISE DATA_FUTURA;
    END IF;    
EXCEPTION 
        WHEN DATA_FUTURA THEN 
        RAISE_APPLICATION_ERROR(-20002,'DATA INSERITA NON POSSIBILE');
END;
\end{minted}
\subsection{Trigger per aggiornare lo stato di un sensore in caso di anomalie gravi}
e un'anomalia ha livello ≥ 4, modificare automaticamente lo stato del sensore in "Guasto" o "Manutenzione".
\begin{minted}{sql}
CREATE OR REPLACE TRIGGER STATO_ANOMALI
AFTER INSERT ON ANOMALIE
FOR EACH ROW
BEGIN
    IF :NEW.LIVELLO >=4 THEN
    UPDATE SENSORI
    SET STATO = 'GUASTO'
    WHERE ID = :NEW.SENSORE;
    END IF;
END;
\end{minted}
\subsection{Trigger per impedire la cancellazione di membri che hanno scritto report}
Evitare la rimozione di un membro se ha creato almeno un report.
\begin{minted}{sql}
CREATE OR REPLACE TRIGGER NO_DELETE_MEMBER
BEFORE DELETE ON MEMBRI
FOR EACH ROW
DECLARE
    COUNTER NUMBER;
BEGIN
    SELECT COUNT(*) INTO COUNTER
    FROM REPORT
    WHERE AUTORE = :OLD.CODICE;
    IF COUNTER > 0 THEN
        RAISE_APPLICATION_ERROR(-20004, 'ERRORE: Impossibile eliminare il membro perché ha scritto uno o più report.');
    END IF;
END;
\end{minted}
\end{description}
%%% Local Variables:
%%% mode: LaTeX
%%% TeX-master: "Tesina"
%%% End:

%%% Local Variables:
%%% mode: LaTeX
%%% TeX-master: "Tesina"
%%% End: