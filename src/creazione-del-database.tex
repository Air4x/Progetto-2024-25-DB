\section{Creazione del database}
%TODO: Da completare
Delineato il modello logico, tramite i comandi DDL possiamo inserire all'interno del data base gli schemi di relazioni associate alle tabelle.
\begin{description}

\subsection{Sensori}
\begin{minted}{sql}
CREATE SEQUENCE sensori_id_seq START WITH 1;
CREATE TABLE SENSORI (
    ID INTEGER DEFAULT sensori_id_seq.nextval NOT NULL,
    STATO VARCHAR(30) NOT NULL,
    DATAINSTALLAZIONE DATE NOT NULL,
    TIPO VARCHAR(30) NOT NULL,
    UTLIMOCONTROLLO DATE,
    LATITUDINE DECIMAL(10,4) NOT NULL,
    LONGITUDINE DECIMAL(10,4) NOT NULL,
    ALTITUDINE DECIMAL(10,4) NOT NULL,
    
    CONSTRAINT PK_ID PRIMARY KEY(ID)
    
    );
\end{minted}

\subsection{Anomalie}
\begin{minted}{sql}
CREATE TABLE ANOMALIE(
    DATA DATE NOT NULL,
    ORA DATE NOT NULL,
    CAUSA VARCHAR(30) NOT NULL,
    LIVELLO INTEGER NOT NULL,
    SENSORE INTEGER NOT NULL,
    CONSTRAINT PK_ANOMALIE PRIMARY KEY (DATA,ORA),
    CONSTRAINT FK_ANOMALIE_SENSORI FOREIGN KEY (SENSORE) REFERENCES SENSORI (ID) ON DELETE CASCADE
    );
\end{minted}

\subsection{Interventi}
\begin{minted}{sql}
CREATE SEQUENCE interventi_codice_seq START WITH 1;
CREATE TABLE INTERVENTI(
    CODICE INTEGER DEFAULT interventi_codice_seq.nextval NOT NULL,
    ESITO VARCHAR(20) NOT NULL,
    CALIBRAZIONE VARCHAR(20) NOT NULL,
    RIPARAZIONE VARCHAR(20) NOT NULL,
    SOSTITUZIONE VARCHAR(20) NOT NULL,
    DATAPREFISSATA DATE NOT NULL,
    
    CONSTRAINT PK_INTERVENTI PRIMARY KEY(CODICE)
    );
\end{minted}

\subsection{Membri}
\begin{minted}{sql}
CREATE SEQUENCE membri_codice_seq START WITH 1;
CREATE TABLE MEMBRI(
    CODICE INTEGER DEFAULT membri_codice_seq.nextval NOT NULL,
    NOME VARCHAR(20) NOT NULL,
    COGNOME VARCHAR(20) NOT NULL,
    RUOLO VARCHAR(20) NOT NULL,
    
    CONSTRAINT PK_MEMBRI PRIMARY KEY(CODICE)
    );
\end{minted}

\subsection{Report}
\begin{minted}{sql}
CREATE TABLE REPORT(
    STATO VARCHAR(30) NOT NULL,
    AUTORE INTEGER NOT NULL,
    
    CONSTRAINT PK_REPORT PRIMARY KEY(STATO),
    CONSTRAINT FK_REPORT_MEMBRI FOREIGN KEY(AUTORE) REFERENCES MEMBRI(CODICE) ON DELETE CASCADE
    );
\end{minted}

\subsection{Missioni}
\begin{minted}{sql}
CREATE SEQUENCE missioni_codice_seq START WITH 1;
CREATE TABLE MISSIONI(
    CODICE INTEGER DEFAULT missioni_codice_seq.nextval NOT NULL,
    DATAFINE DATE NOT NULL,
    OBBIETTIVO VARCHAR(100) NOT NULL,
    DATAINIZIO DATE NOT NULL,
    STATOATTUALE VARCHAR(30) NOT NULL,
    RESOCONTO VARCHAR(30) NOT NULL,
    
    CONSTRAINT PK_MISSIONI PRIMARY KEY (CODICE),
    CONSTRAINT FK_MISSIONI_REPORT FOREIGN KEY (RESOCONTO) REFERENCES REPORT(STATO) ON DELETE CASCADE
);
\end{minted}

\subsection{Rilevazioni}
\begin{minted}{sql}
CREATE TABLE RILEVAZIONI(
    DATA DATE NOT NULL,
    ORA DATE NOT NULL, 
    VALORE INTEGER NOT NULL,
    SENSORE INTEGER NOT NULL,
    
    CONSTRAINT PK_RILEVAZIONI PRIMARY KEY(DATA,ORA),
    CONSTRAINT FK_RILEVAZIONI_SENSORI FOREIGN KEY(SENSORE) REFERENCES SENSORI (ID) ON DELETE CASCADE
)
\end{minted}

\subsection{Robot}
\begin{minted}{sql}
CREATE SEQUENCE robot_codice_seq START WITH 1;
CREATE TABLE ROBOT(
    CODICE INTEGER DEFAULT robot_codice_seq.nextval NOT NULL,
    TIPOLOGIA VARCHAR(30),
    
    CONSTRAINT PK_ROBOT PRIMARY KEY(CODICE)
    );
\end{minted}

\subsection{Risorsa_1}
\begin{minted}{sql}
CREATE TABLE RISORSA_1(
    ROBOT INTEGER NOT NULL,
    MISSIONE INTEGER NOT NULL,
    CONSTRAINT PK_RISORSA_1 PRIMARY KEY(ROBOT,MISSIONE),
    CONSTRAINT FK_RISORSA1_ROBOT FOREIGN KEY(ROBOT) REFERENCES ROBOT(CODICE) ON DELETE CASCADE, 
    CONSTRAINT FK_RISORSA1_MISSIONE FOREIGN KEY(MISSIONE) REFERENCES MISSIONI(CODICE) ON DELETE CASCADE
    );
\end{minted}

\subsection{Risorsa_2}
\begin{minted}{sql}
CREATE TABLE RISORSA_2(
    SENSORE INTEGER NOT NULL,
    MISSIONE INTEGER NOT NULL,
    
    CONSTRAINT PK_RISORSA_2 PRIMARY KEY (SENSORE,MISSIONE),
    CONSTRAINT FK_RISORSA2_SENSORE FOREIGN KEY(SENSORE) REFERENCES SENSORI(ID) ON DELETE CASCADE,
    CONSTRAINT FK_RISORSA2_MISSIONE FOREIGN KEY(MISSIONE) REFERENCES MISSIONI(CODICE) ON DELETE CASCADE
    
    );
\end{minted}
\end{description}


%%% Local Variables:
%%% mode: LaTeX
%%% TeX-master: "Tesina"
%%% End:
