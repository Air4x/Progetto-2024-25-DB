\section{Creazione Schema Relazionale}
In questa sezione ci occupiamo della creazione dello schema relazionale partendo dal
modello E/R in \ref{fig:er-finale}. Per la traduzione da modello E/R a schema relazionale
si seguono le seguenti regole:
\begin{enumarate}
\item Ciascuna \textit{entità} viene trasformata in una relazione,
  dove gli attributi dell'entità diventano gli attributi della
  relazione, con la chiave primaria dell'entità che diventa
  l'identificatore della relazione
\item Le relazioni vengono tradotte in base alla loro cardinalità:
  \begin{description}
  \item[uno a uno] si traducono aggiungendo alle relazioni che traducono le entità dal lato uno gli
    attributi delle associazioni e gli identificatori delle entità lato molti. Questi ultimi saranno soggetti
    ad un vincolo di integrità referenziale con il corrispondente attributo delle entità dal lato molti
  \item[uno a molti] si traducono aggiungendo alla relazione che traduce una delle due entità gli attributi
    dell’associazione, oltre che il suo identificatore che sarà soggetto sia ad un vincolo di unicità
    che di integrità referenziale con il corrispondente attributo dell’altra entità.
  \end{description}
\end{enumarate}

% TODO: Creare schema relazionale
Seguendo queste regole si ottiene il seguente schema:
\begin{itemize}
  % TODO: Completare SENSORE
\item \texttt{ANOMALIE}();
  % TODO: Completare SENSORE
\item \texttt{INTERVENTI}();
  % TODO: Completare SENSORE
\item \texttt{MEMBRI}();
  % TODO: Completare SENSORE
\item \texttt{MISSIONI}(\underline{codice},);
  % TODO: Completare SENSORE
\item \texttt{REPORT}();
  % TODO: Completare SENSORE
\item \texttt{RILEVAZIONI}();
  % TODO: Completare SENSORE
\item \texttt{ROBOT}();
  % TODO: Completare SENSORE
\item \texttt{SENSORI}();
\end{itemize}

%%% Local Variables:
%%% mode: LaTeX
%%% TeX-master: "Tesina"
%%% End:
