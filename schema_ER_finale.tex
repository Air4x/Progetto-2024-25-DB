% TODO: Scrivere discussione dello schema ER finale
\section{Schema E/R finale}
Da una analisi più approfondia si vista la necxessita di introdurre delle nuove entità:
\begin{description}
\item [Anomalia] con cui si va ad indicare un Sensore non funzionante.
\item [Intervento] per la riparazione di un Sensore non funzionante.
\item [Rilevazione] con cui si va indentificare i dati ralcolti dai sensori.
\end{description}
Caratterizate dalle seguenti relazioni:
\begin{description}
\item [Analisi] che va a legare Rivelazioni e Sensore, con cui si va ad indicare i dati racolti dai vari sensori.
\item [Malfunzionamento] che va a legare Sensore ed Anomalia, tramite cui si individuano i sensori non funzionanti.
\item [Risoluzione] che va ad legare Anomalia ed Intervento, azine con cui si va ad riparere un sensore non funzionante.
\end{description}
Nel modello \ref{fig:er-portante} sono riporta le \textbf{cardinalita}specificando il numero minimo e il numero massimo che, le occorrenze, possono assumere in ciascuna associazione, rispetto alle entità. Le cardinalità delle relazioni sono le seguenti: 
\begin{itemize}
\item \textbf{Partecipazione}: di tipo \textbf{molti a molti (N,N)}; un membro dell'equipaggio può essere stato selezionato per la partecipazione di diverse missioni spaziali, mentre una missione richiede almeno un membro dell'equipaggio.
\item \textbf{Stesura}: di tipo \textbf{molti a molti (N,N)}; un report può essere scritto da molteplici membri dell'equipaggio, mentre un membro dell'equipaggio può comporre diversi report.
\item \textbf{Report}: di tipo \textbf{uno a molti (1,N)}; indica l'appartenenza di un report ad un'unica missione, quando invece una missione è composta da diversi report che informano dello stato della missione man mano.
\item \textbf{Risorsa-1} e \textbf{Risorsa-2}:: entrmbe di tipo \textbf{molti a molti (N,N)};in quanto una missione ha la possibilità di utilizzare molteplici sensori, quando un sensore viene ripiegato per una molteplicità di missioni. Analogalmente la relazione è identica per l'utilizzo dei robot nelle missioni.
\item \textbf{Malfunzionamento}: di tipo \textbf{molti a molti (1,N)}; in quanto quando si presenta un'anomalia, essa si riferisce ad un determinato sensore. Invece per i sensori vi è la possibilità di trovare diverse anomalie.
\item \textbf{Risoluzione}: di tipo \textbf{uno a molti (1,N)} quando si presenta un'anomalia, un intervento può risolverla. Un intervento può essere applicato su diverse anomalie quando hanno la stessa causa;.
\item \textbf{Eseguito}: di tipo \textbf{molti a molti (N,N)}; un membro dell'equipaggio può eseguire innumerevoli interventi. Un intervento ha la possibilità di essere eseguito, sia da nessun membro, che da molteplici membri.
\item \textbf{Analisi}: di tipo \textbf{uno a molti (1,N)}; una rilevazione viene analizzata da un determinato sensore, il sensore invece effettua diverse rilevazioni.
\end{itemize}